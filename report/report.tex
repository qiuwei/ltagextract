%        File: task-oriented_parser_evaluation.tex
%     Created: 周五 三月 02 04:00 下午 2012 W EST
% Last Change: 周五 三月 02 04:00 下午 2012 W EST
%
\documentclass[a4paper]{article}
%\usepackage{natbib}
%\usepackage{biblatex}
\usepackage{hyperref}
\usepackage{longtable}
\usepackage[style=authoryear,natbib=true]{biblatex}
\bibliography{D:/Research/bib/Parsing}
\usepackage{graphicx}
%\usepackage{gb4e}
\usepackage{lingmacros}
\title{Lexicalized Tree Adjoining Grammar Extraction For KBGEN}
\author{Wei Qiu \and Jiri Marsik}
%\date{}
\begin{document}
\maketitle
\begin{abstract}
\end{abstract}
\section{Introduction}
\section{Semantic Annotation}
\subsection{Annotation guidelines}
\subsection{Problems}


\section{Preprocessing}

\subsection{Parsing with head information}

Our plan is to extract the elementary trees for a TAG from a treebank.
Since our corpus lacks phrase structure information, we ran an
off-the-shelf parser on it. The parser that we used for this task is
the Stanford parser.

At the outset, we had plans to use both the phrase structure and
dependency parses that the Stanford parser could produce. Previous
research has shown that having a dependency parse facilitates the task
of determining headedness, which is usually a subtask of elementary
tree extraction.

Finally, we have found a simpler way of getting the head information
directly in the phrase structure parse trees through a command line
switch in the Stanford parser.


Furthermore, the Stanford parser offers two pre-trained models for
parsing English, an unlexicalized one and a lexicalized one. We have
done the parsing using both of them and try to review their
performance. Their outputs were significantly different and both
performed their own idiosyncratic errors. We judged which of these
errors would likely be more harmful to the following grammar
extraction process and decided to stick with the output of the
unlexicalized parser (though the result of the unlexicalized parsing
almost as just as bad as that of the lexicalized one).

\subsection{Conjunction resolving}

\subsection{Normalizing}


\section{TAG extraction}
\section{Semantic roles alignment}
\section{Implementation and result}
data format
\section{Conclusion \& Future work}
\subsection{Future work}
\begin{itemize}
    \item workaround with parsing errors.
    \item more robust extraction algorithms
    \item more robust alignment algorithms
\end{itemize}
\end{document}
